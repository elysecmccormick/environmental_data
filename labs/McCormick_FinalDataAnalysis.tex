% Options for packages loaded elsewhere
\PassOptionsToPackage{unicode}{hyperref}
\PassOptionsToPackage{hyphens}{url}
%
\documentclass[
]{article}
\usepackage{amsmath,amssymb}
\usepackage{lmodern}
\usepackage{iftex}
\ifPDFTeX
  \usepackage[T1]{fontenc}
  \usepackage[utf8]{inputenc}
  \usepackage{textcomp} % provide euro and other symbols
\else % if luatex or xetex
  \usepackage{unicode-math}
  \defaultfontfeatures{Scale=MatchLowercase}
  \defaultfontfeatures[\rmfamily]{Ligatures=TeX,Scale=1}
\fi
% Use upquote if available, for straight quotes in verbatim environments
\IfFileExists{upquote.sty}{\usepackage{upquote}}{}
\IfFileExists{microtype.sty}{% use microtype if available
  \usepackage[]{microtype}
  \UseMicrotypeSet[protrusion]{basicmath} % disable protrusion for tt fonts
}{}
\makeatletter
\@ifundefined{KOMAClassName}{% if non-KOMA class
  \IfFileExists{parskip.sty}{%
    \usepackage{parskip}
  }{% else
    \setlength{\parindent}{0pt}
    \setlength{\parskip}{6pt plus 2pt minus 1pt}}
}{% if KOMA class
  \KOMAoptions{parskip=half}}
\makeatother
\usepackage{xcolor}
\usepackage[margin=1in]{geometry}
\usepackage{graphicx}
\makeatletter
\def\maxwidth{\ifdim\Gin@nat@width>\linewidth\linewidth\else\Gin@nat@width\fi}
\def\maxheight{\ifdim\Gin@nat@height>\textheight\textheight\else\Gin@nat@height\fi}
\makeatother
% Scale images if necessary, so that they will not overflow the page
% margins by default, and it is still possible to overwrite the defaults
% using explicit options in \includegraphics[width, height, ...]{}
\setkeys{Gin}{width=\maxwidth,height=\maxheight,keepaspectratio}
% Set default figure placement to htbp
\makeatletter
\def\fps@figure{htbp}
\makeatother
\setlength{\emergencystretch}{3em} % prevent overfull lines
\providecommand{\tightlist}{%
  \setlength{\itemsep}{0pt}\setlength{\parskip}{0pt}}
\setcounter{secnumdepth}{-\maxdimen} % remove section numbering
\ifLuaTeX
  \usepackage{selnolig}  % disable illegal ligatures
\fi
\IfFileExists{bookmark.sty}{\usepackage{bookmark}}{\usepackage{hyperref}}
\IfFileExists{xurl.sty}{\usepackage{xurl}}{} % add URL line breaks if available
\urlstyle{same} % disable monospaced font for URLs
\hypersetup{
  pdftitle={Final Data Analysis},
  pdfauthor={Elyse McCormick},
  hidelinks,
  pdfcreator={LaTeX via pandoc}}

\title{Final Data Analysis}
\author{Elyse McCormick}
\date{2022-12-14}

\begin{document}
\maketitle

\textbf{Q1 (2 pts.): Qualitatively describe the relationship between
body mass and length.Does the relationship seem linear, curved,
nonexistent?}

The relationship between body mass and body length seems mostly linear,
but left-skewed with an array of outliers.

\textbf{Q2 (2 pts.): Qualitatively describe the shapes of the
histograms. Do the data appear normally-distributed? Explain why or why
not.Explain why we care (or not) whether the data are normally
distributed.}

Body mass seems relatively normally distributed around 40 g, but the
middle of the curve is much higher than the rest of the dataset,
indicating it's not a perfectly normal distribution.

Body length is not normally distributed, with what is almost a peak at
100 cm, and very little spread, indicating leptokurtosis. This also
indicates it is not a normal distribution.

Normally distributed data means that we can run parametric tests on the
data. Additionally, normal distributions are

\textbf{Q3 (2 pts.): Using both the histograms and normality tests, do
you think the (unconditioned) body masses and body length are
normally-distributed?Make sure you contrast your visual assessment of
normality to the results of the numerical normality tests.}

Between both the histograms and the Shapiro-Wilk tests, I don't think
these data are normally distributed. As described in question 2, neither
histogram is perfectly normally distributed. Similarly, the p values of
the Shapiro-Wilk tests showed p \textless{} 0.05 (body mass: p =
4.33e-05, body length: p = 2.2e-16), indicating that the data are not
normally distributed.

\textbf{Q4 (2 pts.): Examine the three conditional boxplots.Describe any
graphical evidence you see for body mass differences based on species
and/or sex.}

When looking by species, the conditional boxplot shows that D. dorsalis
has a slightly higher mass. When looking at the conditional boxplot for
sex, the boxplots show no difference. When looking at the conditional
boxplot for both sex and species, it indicates that in both species,
females seem to have a higher weight than male.

\textbf{Q5 (2 pts.): What do you conclude about residual normality based
on the numerical and graphical diagnostics?}

The data appear to be non-normal from both the histograms and the
Shapiro-Wilk tests. However, the graphical diagnostics would lead me to
believe the data are slightly more normal than the Shapiro Tests
indicate. The Shapiro-Wilk tests have extremely low p values, while some
of the histograms have mild to extreme kurtosis. The first set of
residuals is completely leptokurtotic, while the distribution improves
with every subsequent model.

\textbf{Q6 (1 pt.): Are violations of the normality assumption equally
severe for all the models?}

Graphically no, but the p values from the Shapiro-Wilk tests indicate
that all 5 models are very non-normal, with p values well below p =
0.0001.

\textbf{Q7 (2 pts.): What is the magnitude of the mass/length
relationship?}

The magnitude of the mass/length relationship is the mean estimate,
which in model 1 is 0.8755.

\textbf{Q8 (2 pts.): What is the expected body length of an animal that
weighs 100g?}

The expected body length is 76.12 cm.

\textbf{Q9 (2 pts.): What is the expected body length of an animal that
weighs 0g?}

The expected body length is 0 cm.

\textbf{Q10 (1 pt.): What is the base level for sex?}

Body mass

\textbf{Q11 (1 pt.): What is the base level for binomial?}

Body Mass

\textbf{Q12 (1 pt.): Which sex is heavier? How do you know?}

Females are heavier, based on the conditional boxplots. Based on the
available output, without post-hoc tests I can't determine this another
way.

\textbf{Q13 (1 pt.): Which species is heavier? How do you know?}

Delomys dorsalis is heavier, based on the conditional boxplots. From the
available output, without post-hoc tests I can't determine this another
way.

\textbf{Q14 (1 pt.): Are sex and species significant predictors for body
mass?}

Yes, sex and species are significant predictors for body mass.

\textbf{Q15 (1 pt.): Is there a significant interaction?}

No, there is not a significant factorial interaction between them, but
there is an additive effect between species and sex.

\textbf{Q16 (2 pts.): Examine the p-values for the main effects (sex and
species) in all four of the ANOVA tables. Does the significance level of
either main effect change very much among the different models?}

No, the significance level stays extremely similar throughout the
models.

\textbf{Q17 (1 pt.): Which two models have the lowest AIC?}

Models fit1 and fit4 have the lowest AIC scores.

\textbf{Q18 (4 pts.): Which of the two models with lowest AIC scores
would you select? Explain your decision based on model fit and the
complexity/understanding tradeoff.}

I would choose fit4 as the best model choice. Though it had the second
to lowest AIC score (fit1 was lower), fit4 incorporates more complexity
while still showing what the best model fit is. This allows us to see
interactions between both species and sex within these mice, while fit1
only can analyze body length and body mass. This complexity gives better
insight into the biology of the interaction, making fit4 a good model to
select. This is a good use of the trade off, as the other three models
had varying levels of complexity but higher AIC scores, indicating the
fit was not ideal for other models.

\end{document}
